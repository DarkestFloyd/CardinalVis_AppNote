\documentclass{bioinfo}
\copyrightyear{2015} \pubyear{2015}

\access{CardinalVis}
\appnotes{Manuscript Category}

\begin{document}
\firstpage{1}

\subtitle{Subject Section}

\title[short Title]{CardinalVis}
\author[Sample \textit{et~al}.]{Kylie Bemis\,$^{\text{\sfb 1,}*}$ and Nischal Mahaveer Chand\,$^{\text{\sfb 1}}$}
\address{$^{\text{\sf 1}}$Khoury College, Northeastern University, Boston, 02115, USA}

\corresp{$^\ast$To whom correspondence should be addressed.}

\history{Received on XXXXX; revised on XXXXX; accepted on XXXXX}

\editor{Associate Editor: XXXXXXX}

\abstract{\textbf{Motivation:} Cardinal is an R package for statistical analysis of mass spectrometry-based imaging (MSI) experiments  of  biological  samples  such  as  tissues. Cardinal allows users to analyse, visualize, and perform statistical analysis on MSI experiments. However, Cardinal is purely a command line tool with no interactive graphical user interface (GUI). Visualization are generated on demand but provide very limited interactivity. Thus motivating CardinalVis, a interactive GUI to act as an interfacet to Cardinal. \\
\textbf{Results:} 
CardinalVis is an R ``Shiny'' dashboard that aims to provide much of the functionality of Cardinal to the user without the need to use the command line, thus enabling researchers to quickly understand their data by enabling rapid exploration of relevant ion images and associate mass spectra. \\
\textbf{Availability:} Text  Text Text Text Text Text Text Text Text Text  Text Text Text Text
Text Text Text Text Text Text Text Text Text Text Text Text Text Text  Text\\
\textbf{Contact:} \href{name@bio.com}{name@bio.com}\\
\textbf{Supplementary information:} Supplementary data are available at \textit{Bioinformatics}
online.}

\maketitle

\section{Introduction}

Mass spectrometry imaging (MSI) is a powerful tool that enables untargeted investigations into the spatial distribution of molecular species in a variety of samples. The combination of information gained from mass spectrometry (MS) and visualization of spatial distributions in thin sample sections makes this a valuable chemical analysis tool for biological specimen characterization. \\
Recent advancements in technology and research have improved the quality of MSI images collected, allowing for higher acquisition speeds and enhanced spatial resolution, overall improving throughput and depth. But these have made MSI data difficult to process due to large file sizes with high-dimensional spatial data. These are directly reflected in the analysis phase of a study, where expensive proprietary software and hardware are required. In particular, visualization of MSI experiments is a vital part of exploratory analysis, but can be challenging due to the extremely high-dimensional nature of the data, and the need to simultaneously visualize linked mass spectra and associated ion images (Ion images show the relative distribution of molecules at a particular mass-to-charge (m/z) value). \\

\section{Implementation}

Text Text Text Text Text Text  Text Text Text Text Text Text Text
Text Text  Text Text Text Text Text Text.
Figure~2\vphantom{\ref{fig:02}} shows that the above method  Text
Text Text Text  Text Text Text Text Text Text  Text Text.
\citealp{Boffelli03} might want to know about  text text text text
Text Text Text Text Text Text  Text Text Text Text Text Text Text
Text Text  Text Text Text Text Text Text.
Figure~2\vphantom{\ref{fig:02}} shows that the above method  Text
Text Text Text  Text Text Text Text Text Text  Text Text.
\citealp{Boffelli03} might want to know about  text text text text
Text Text Text Text Text Text Text Text Text Text.

\section*{Acknowledgements}

I would like to thank Dr. Kylie Bemis and Khoury College of Computer Sciences for providing me with the opportunity 
to work on this project.

\begin{thebibliography}{}

% \bibitem[Bofelli {\it et~al}., 2000]{Boffelli03}
% Bofelli,F., Name2, Name3 (2003) Article title, {\it Journal Name}, {\bf 199}, 133-154.

% \bibitem[Bag {\it et~al}., 2001]{Bag01}
% Bag,M., Name2, Name3 (2001) Article title, {\it Journal Name}, {\bf 99}, 33-54.

% \bibitem[Yoo \textit{et~al}., 2003]{Yoo03}
% Yoo,M.S. \textit{et~al}. (2003) Oxidative stress regulated genes
% in nigral dopaminergic neurnol cell: correlation with the known
% pathology in Parkinson's disease. \textit{Brain Res. Mol. Brain
% Res.}, \textbf{110}(Suppl. 1), 76--84.

% \bibitem[Lehmann, 1986]{Leh86}
% Lehmann,E.L. (1986) Chapter title. \textit{Book Title}. Vol.~1, 2nd edn. Springer-Verlag, New York.

% \bibitem[Crenshaw and Jones, 2003]{Cre03}
% Crenshaw, B.,III, and Jones, W.B.,Jr (2003) The future of clinical
% cancer management: one tumor, one chip. \textit{Bioinformatics},
% doi:10.1093/bioinformatics/btn000.

% \bibitem[Auhtor \textit{et~al}. (2000)]{Aut00}
% Auhtor,A.B. \textit{et~al}. (2000) Chapter title. In Smith, A.C.
% (ed.), \textit{Book Title}, 2nd edn. Publisher, Location, Vol. 1, pp.
% ???--???.

% \bibitem[Bardet, 1920]{Bar20}
% Bardet, G. (1920) Sur un syndrome d'obesite infantile avec
% polydactylie et retinite pigmentaire (contribution a l'etude des
% formes cliniques de l'obesite hypophysaire). PhD Thesis, name of
% institution, Paris, France.

\end{thebibliography}
\end{document}
